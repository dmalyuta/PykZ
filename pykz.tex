%%%%%%%%%%%%%%%%%%%%%%%%%%%%%%%%%%%%%%%%%%%%%%%%%%%%%%%%%%%%%%%%%%%%%%%%
%
% PykZ
%
% A package that combines the speed of Python for numerical calculation and the
% drawing capabilities of TikZ, to make publication-quality drawing faster.
%
% Some conventions:
%
%   \@pik@<name> : PykZ internal key <name>
%   \@piv@<name> : PykZ internal variable <name>
%   \@pic@<name> : PykZ internal command <name>
%   @pie@<name> : PykZ internal environment <name>
%
% Danylo Malyuta, 2020
% 
%%%%%%%%%%%%%%%%%%%%%%%%%%%%%%%%%%%%%%%%%%%%%%%%%%%%%%%%%%%%%%%%%%%%%%%%

\makeatletter

%%%%%%%%%%%%%%%%%%%%%%%%%%%%%%%
% Load packages
%%%%%%%%%%%%%%%%%%%%%%%%%%%%%%%

\usepackage{tikz}
\usepackage{tikz-3dplot}
\usepackage{pythontex}

%%%%%%%%%%%%%%%%%%%%%%%%%%%%%%%
% Load TikZ libraries
%%%%%%%%%%%%%%%%%%%%%%%%%%%%%%%

\usetikzlibrary{arrows}
\usetikzlibrary{intersections}
\usetikzlibrary{decorations.markings}
\usetikzlibrary{math}
\usetikzlibrary{shapes}
% \usetikzlibrary{calc}
% \usetikzlibrary{tikzmark}
% \usetikzlibrary{positioning}
% \usetikzlibrary{scopes}
% \usetikzlibrary{shapes.multipart}
% \usetikzlibrary{shapes.geometric}
% \usetikzlibrary{bending}
% \usetikzlibrary{decorations.text}

%%%%%%%%%%%%%%%%%%%%%%%%%%%%%%%
% General functions
%%%%%%%%%%%%%%%%%%%%%%%%%%%%%%%

\gdef\@piv@python@delay{0.5} % Startup wait for Python server

\define@key{@pik@keys}{relpath}{\def\@piv@relpath{#1}}
\newcommand{\pykzinit}[1][]{
  % ---------------------------------------------------------
  %
  % Initialization of PykZ.
  % Starts the Python background server process.
  %
  % ---------------------------------------------------------
  \setkeys{@pik@keys}{relpath=.,#1}
  \xdef\@piv@relpath{\@piv@relpath}
  % Start a background Python server
\begin{@pie@shell}
curdir=$(pwd) &&
cd \@piv@relpath &&
python server.py &
\end{@pie@shell}
  % Allow some start-up time for Python
\begin{@pie@shell}
sleep \@piv@python@delay
\end{@pie@shell}
}

\NewEnviron{@pie@shell}{
  \immediate\write18{\BODY}
}

%%%%%%%%%%%%%%%%%%%%%%%%%%%%%%%
% Math library
%%%%%%%%%%%%%%%%%%%%%%%%%%%%%%%

\newcommand{\pykzmath}[1]{
  % ---------------------------------------------------------
  %
  % Send math command to Python, and store result in \out.
  %
  % ---------------------------------------------------------  
\begin{@pie@shell}
curdir=$(pwd) &&
cd \@piv@relpath &&
./client "#1" &&
mv result.tex $curdir
\end{@pie@shell}
  \input{result.tex}
}

\makeatother

%%% Local Variables:
%%% mode: latex
%%% TeX-master: t
%%% End:
