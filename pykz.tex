%%%%%%%%%%%%%%%%%%%%%%%%%%%%%%%%%%%%%%%%%%%%%%%%%%%%%%%%%%%%%%%%%%%%%%%%
%
% PykZ
%
% A package that combines the speed of Python for numerical calculation and the
% drawing capabilities of TikZ, to make publication-quality drawing faster.
%
% Some conventions:
%
%   \@pik@<name> : PykZ internal key <name>
%   \@piv@<name> : PykZ internal variable <name>
%   \@pic@<name> : PykZ internal command <name>
%   @pie@<name> : PykZ internal environment <name>
%
% Danylo Malyuta, 2020
%
%%%%%%%%%%%%%%%%%%%%%%%%%%%%%%%%%%%%%%%%%%%%%%%%%%%%%%%%%%%%%%%%%%%%%%%%

\makeatletter

%%%%%%%%%%%%%%%%%%%%%%%%%%%%%%%
% Load packages
%%%%%%%%%%%%%%%%%%%%%%%%%%%%%%%

\usepackage{filecontents}
\usepackage{fancyvrb}

%%%%%%%%%%%%%%%%%%%%%%%%%%%%%%%
% General functions
%%%%%%%%%%%%%%%%%%%%%%%%%%%%%%%

\gdef\@piv@python@delay{0.5} % Startup wait for Python server

\newcommand{\pykzinit}[1]{
  % ---------------------------------------------------------
  %
  % Initialization of PykZ.
  % Starts the Python background server process.
  % Must be called before doing anything that uses PykZ.
  %
  % ---------------------------------------------------------
  \xdef\@piv@relpath{#1}
  % Start a background Python server
\begin{@pie@shell}
curdir=$(pwd) $() &&
cd \@piv@relpath &&
python server.py &
\end{@pie@shell}
  % Allow some start-up time for Python
\begin{@pie@shell}
sleep \@piv@python@delay
\end{@pie@shell}
  % Determine path to directory where latex_common resides
  \pykzmathinline[@piv@pypath]{'\@piv@relpath'[:'\@piv@relpath'.find('latex_common')]}
  % Include mathematical plotting functionality
  %%%%%%%%%%%%%%%%%%%%%%%%%%%%%%%%%%%%%%%%%%%%%%%%%%%%%%%%%%%%%%%%%%%%%%%%
%
% Plotting with PykZ
%
% A package that provides facilities for making plots of mathematical functions
% using PykZ.
%
% Some conventions:
%
%   \@pyplot@<name> : PykZ plotting variable
%   \pyplot@<name> : PykZ plotting function
%   \pyplot@internal@<name> : PykZ plotting internal function, don't touch!
%
% Danylo Malyuta, 2021
%
%%%%%%%%%%%%%%%%%%%%%%%%%%%%%%%%%%%%%%%%%%%%%%%%%%%%%%%%%%%%%%%%%%%%%%%%

\makeatletter

%%%%%%%%%%%%%%%%%%%%%%%%%%%%%%%%%%%%%%%
% Working with layers
%%%%%%%%%%%%%%%%%%%%%%%%%%%%%%%%%%%%%%%

\def\@pykz@layers{main}

% Add a new layer behind all current layers
\newcommand{\pykz@add@layer@below}[1]{
  \pgfdeclarelayer{#1}
  \xdef\@pykz@layers{#1,\@pykz@layers}
  \pgfsetlayers{\@pykz@layers}
}

% Add a new layer in front of all current layers
\newcommand{\pykz@add@layer@above}[1]{
  \pgfdeclarelayer{#1}
  \xdef\@pykz@layers{\@pykz@layers,#1}
  \pgfsetlayers{\@pykz@layers}
}

% Define a layer "in front"
\pykz@add@layer@above{pyplot fg}

%%%%%%%%%%%%%%%%%%%%%%%%%%%%%%%%%%%%%%%
% Plotting variables
%%%%%%%%%%%%%%%%%%%%%%%%%%%%%%%%%%%%%%%

\def\@pyplot@arrow{Stealth[round]}
\def\@pyplot@unit{cm}
\def\@pyplot@clip{True}
\def\@pyplot@plotarea@x@size{2}
\def\@pyplot@plotarea@y@size{2}
\def\@pyplot@plotarea@xmin{-1}
\def\@pyplot@plotarea@xcenter{0}
\def\@pyplot@plotarea@xmax{1}
\def\@pyplot@plotarea@ymin{-1}
\def\@pyplot@plotarea@ycenter{0}
\def\@pyplot@plotarea@ymax{1}
\def\@pyplot@axarea@pad@left{0}
\def\@pyplot@axarea@pad@right{0}
\def\@pyplot@axarea@pad@top{0}
\def\@pyplot@axarea@pad@bottom{0}
\def\@pyplot@map@precision{4}
\def\@pyplot@python@maxchar{1015}
\def\@pyplot@background@color{white}

%%%%%%%%%%%%%%%%%%%%%%%%%%%%%%%%%%%%%%%
% Drawing styles
%%%%%%%%%%%%%%%%%%%%%%%%%%%%%%%%%%%%%%%

\tikzset{
  pyplot axis/.style={
    -\@pyplot@arrow,
    black
  },
  pyplot axis label/.style={},
  pyplot plot area debug/.style={
    dashed,
    opacity=0.5
  },
  pyplot mark position/.style args={#1(#2)}{
    postaction={
      decorate,
      decoration={
        markings,
        mark=at position #1 with \coordinate (#2);
      }
    }
  },
  pyplot major tick/.style={
    line width=0.3
  },
  pyplot major tick label/.style={
    inner sep=0.03\@pyplot@unit
  },
  pyplot tick background/.style={
    \@pyplot@background@color,
    rounded corners=\@pyplot@ticks@size/2
  }
}

%%%%%%%%%%%%%%%%%%%%%%%%%%%%%%%%%%%%%%%
% Public functions
%%%%%%%%%%%%%%%%%%%%%%%%%%%%%%%%%%%%%%%

% Define the units. In all commands, leave numbers dimensionles!
\newcommand{\pyplot@set@unit}[1]{
  \xdef\@pyplot@unit{#1}
}

% Define the background color
\newcommand{\pyplot@set@bg}[1]{
  \xdef\@pyplot@background@color{#1}
}

% Set whether the plotted function coordinates are clipped before being
% returned to the user
\newcommand{\pyplot@set@clip}[1]{
  \xdef\@pyplot@clip{#1}
}

% Define the plot size
\newcommand{\pyplot@set@size}[2]{
  \xdef\@pyplot@plotarea@x@size{#1*1\@pyplot@unit}
  \xdef\@pyplot@plotarea@y@size{#2*1\@pyplot@unit}
}

% Define the x-axis limits
\define@key{@pyplot@keys}{center}{\def\@pyplot@plotarea@center{#1}}
\newcommand{\pyplot@set@xlim}[3][]{
  \setkeys{@pyplot@keys}{center=\@pyplot@plotarea@xcenter,#1}%
  \xdef\@pyplot@plotarea@xcenter{\@pyplot@plotarea@center}
  \xdef\@pyplot@plotarea@xmin{#2}
  \xdef\@pyplot@plotarea@xmax{#3}
}

% Define the y-axis limits
\newcommand{\pyplot@set@ylim}[3][]{
  \setkeys{@pyplot@keys}{center=\@pyplot@plotarea@ycenter,#1}%
  \xdef\@pyplot@plotarea@ycenter{\@pyplot@plotarea@center}
  \xdef\@pyplot@plotarea@ymin{#2}
  \xdef\@pyplot@plotarea@ymax{#3}
}

% Define the padding around the plot area
\define@key{@pyplot@keys}{left}{\def\@pyplot@axarea@pad@left@key{#1}}
\define@key{@pyplot@keys}{right}{\def\@pyplot@axarea@pad@right@key{#1}}
\define@key{@pyplot@keys}{top}{\def\@pyplot@axarea@pad@top@key{#1}}
\define@key{@pyplot@keys}{bottom}{\def\@pyplot@axarea@pad@bottom@key{#1}}
\newcommand{\pyplot@set@padding}[1]{
  \setkeys{@pyplot@keys}{left=\@pyplot@axarea@pad@left,#1}%
  \setkeys{@pyplot@keys}{right=\@pyplot@axarea@pad@right,#1}%
  \setkeys{@pyplot@keys}{top=\@pyplot@axarea@pad@top,#1}%
  \setkeys{@pyplot@keys}{bottom=\@pyplot@axarea@pad@bottom,#1}
  \xdef\@pyplot@axarea@pad@left{\@pyplot@axarea@pad@left@key*1\@pyplot@unit}
  \xdef\@pyplot@axarea@pad@right{\@pyplot@axarea@pad@right@key*1\@pyplot@unit}
  \xdef\@pyplot@axarea@pad@top{\@pyplot@axarea@pad@top@key*1\@pyplot@unit}
  \xdef\@pyplot@axarea@pad@bottom{\@pyplot@axarea@pad@bottom@key*1\@pyplot@unit}
}

% Draw the axes
\define@key{@pyplot@keys}{showarea}{\def\@pyplot@show@area@debug{#1}}
\define@key{@pyplot@keys}{yaxloc}{\def\@pyplot@y@axis@draw@location{#1}}
\newcommand{\pyplot@draw@axes}[1][]{
  \setkeys{@pyplot@keys}{showarea=false,#1}
  \setkeys{@pyplot@keys}{yaxloc=center,#1}

  \coordinate (pyplot origin) at (0,0);

  % ..:: x-axis ::..

  \pgfmathsetmacro{\@pyplot@negative@frac}{%
    (\@pyplot@plotarea@xcenter-\@pyplot@plotarea@xmin)/
    (\@pyplot@plotarea@xmax-\@pyplot@plotarea@xmin)}
  \pgfmathsetmacro{\@pyplot@positive@frac}{%
    1-\@pyplot@negative@frac}

  \coordinate (pyplot plotarea left) at ($(pyplot origin)+
  (-\@pyplot@negative@frac*\@pyplot@plotarea@x@size,0)$);

  \coordinate (pyplot plotarea right) at ($(pyplot origin)+
  (\@pyplot@positive@frac*\@pyplot@plotarea@x@size,0)$);

  \coordinate (pyplot axarea left) at ($(pyplot plotarea left)+
  (-\@pyplot@axarea@pad@left,0)$);

  \coordinate (pyplot axarea right) at ($(pyplot plotarea right)+
  (\@pyplot@axarea@pad@right,0)$);

  \path let
  \p1=(pyplot plotarea left),
  \p2=(pyplot plotarea right) in \pgfextra{
    \xdef\@pyplot@plotarea@tikz@xmin{\x1}
    \xdef\@pyplot@plotarea@tikz@xmax{\x2}
    \dimensionalize[unit=\@pyplot@unit]{\@pyplot@plotarea@tikz@xmin,
      \@pyplot@plotarea@tikz@xmax}
    \dimensionalize[unit=]{\@pyplot@plotarea@tikz@xmin,
      \@pyplot@plotarea@tikz@xmax}
  };

  \draw[pyplot axis] (pyplot axarea left) -- (pyplot axarea right)
  node[pos=0](pyplot x base){} node[pos=1](pyplot x tip){};

  % ..:: y-axis ::..

  \ifthenelse{\equal{\@pyplot@y@axis@draw@location}{center}}{
    \gdef\@pyplot@y@axis@draw@ref@point{pyplot origin}
  }{
    \gdef\@pyplot@y@axis@draw@ref@point{pyplot axarea left}
  }

  \pgfmathsetmacro{\@pyplot@negative@frac}{%
    (\@pyplot@plotarea@ycenter-\@pyplot@plotarea@ymin)/
    (\@pyplot@plotarea@ymax-\@pyplot@plotarea@ymin)}
  \pgfmathsetmacro{\@pyplot@positive@frac}{%
    1-\@pyplot@negative@frac}

  \coordinate (pyplot plotarea bottom) at ($(\@pyplot@y@axis@draw@ref@point)+
  (0,-\@pyplot@negative@frac*\@pyplot@plotarea@y@size)$);

  \coordinate (pyplot plotarea top) at ($(\@pyplot@y@axis@draw@ref@point)+
  (0,\@pyplot@positive@frac*\@pyplot@plotarea@y@size)$);

  \coordinate (pyplot axarea bottom) at ($(pyplot plotarea bottom)+
  (0,-\@pyplot@axarea@pad@bottom)$);

  \coordinate (pyplot axarea top) at ($(pyplot plotarea top)+
  (0,\@pyplot@axarea@pad@top)$);

  \path let
  \p1=(pyplot plotarea bottom),
  \p2=(pyplot plotarea top) in \pgfextra{
    \xdef\@pyplot@plotarea@tikz@ymin{\y1}
    \xdef\@pyplot@plotarea@tikz@ymax{\y2}
    \dimensionalize[unit=\@pyplot@unit]{\@pyplot@plotarea@tikz@ymin,
      \@pyplot@plotarea@tikz@ymax}
    \dimensionalize[unit=]{\@pyplot@plotarea@tikz@ymin,
      \@pyplot@plotarea@tikz@ymax}
  };

  \draw[pyplot axis] (pyplot axarea bottom) -- (pyplot axarea top)
  node[pos=0](pyplot y base){} node[pos=1](pyplot y tip){};

  % ..:: Other handy variables while we're here ::..

  \coordinate (pyplot plotarea bottom left) at
  (pyplot plotarea left|-pyplot plotarea bottom);

  \coordinate (pyplot plotarea top right) at
  (pyplot plotarea top-|pyplot plotarea right);

  \coordinate (pyplot axarea bottom left) at
  (pyplot axarea left|-pyplot axarea bottom);

  \coordinate (pyplot axarea top right) at
  (pyplot axarea top-|pyplot axarea right);

  \ifthenelse{\equal{\@pyplot@show@area@debug}{true}}{
    \draw[pyplot plot area debug]
    (pyplot plotarea bottom left) rectangle (pyplot plotarea top right);
    \draw[pyplot plot area debug]
    (pyplot axarea bottom left) rectangle (pyplot axarea top right);
  }{}

  \pyplot@generate@scaling@function
}

% Generate a command that produce node scaling to required text height
\newcommand{\pyplot@generate@scaling@function}{
  \coordinate (pyplot plotarea middle) at
  ($(pyplot plotarea bottom left)!0.5!(pyplot plotarea top right)$);

  \begin{scope}
    % Clip so that the invisible node doesn't take up any more space than
    % the actual plot
    \clip (pyplot plotarea bottom left) rectangle (pyplot plotarea top right);

    % Invisible normal font example label, just for its size
    \node (pyplot tmp) at (pyplot plotarea middle)
    {\phantom{$123$}};

    % Compute the scaling
    \path let
    \p1=(pyplot tmp.south),
    \p2=(pyplot tmp.north) in \pgfextra{
      \tikzmath{\@pyplot@internal@text@normal@height = \y2-\y1;}
      \dimensionalize[unit=\@pyplot@unit]{\@pyplot@internal@text@normal@height}
      \xdef\@pyplot@internal@text@normal@height{
        \@pyplot@internal@text@normal@height}
    };
  \end{scope}

  \newcommand{\pyplot@internal@compute@scaling}[1]{
    \tikzmath{\@pyplot@label@scale=##1\@pyplot@unit/
      \@pyplot@internal@text@normal@height;}
    \xdef\@pyplot@label@scale{\@pyplot@label@scale}
  }
}

% Add axis ticks
\define@key{@pyplot@keys}{axis}{\def\@pyplot@tick@axis{#1}}
\define@key{@pyplot@keys}{internal axis}{\def\@pyplot@tick@internal@axis{#1}}
\define@key{@pyplot@keys}{min}{\def\@pyplot@ticks@min{#1}}
\define@key{@pyplot@keys}{max}{\def\@pyplot@ticks@max{#1}}
\define@key{@pyplot@keys}{count}{\def\@pyplot@ticks@count{#1}}
\define@key{@pyplot@keys}{tick size}{\def\@pyplot@ticks@size{#1}}
\define@key{@pyplot@keys}{label size}{\def\@pyplot@tick@label@size{#1}}
\define@key{@pyplot@keys}{label offset}{\def\@pyplot@tick@label@offset{#1}}
\define@key{@pyplot@keys}{format}{\def\@pyplot@label@format{#1}}
% The main command to add ticks
\newcommand{\pyplot@add@ticks}[1]{
  \setkeys{@pyplot@keys}{axis=,#1}
  \pykzmathinline[@pyplot@num@ax]{len('\@pyplot@tick@axis')}
  \pgfmathparse{\@pyplot@num@ax>=1 ? 1 : 0}
  \ifthenelse{\pgfmathresult>0}{
    \foreach \@ax@i in {1,...,\@pyplot@num@ax} {
      \pykzmathinline[@pyplot@which@ax]{'\@pyplot@tick@axis'[\@ax@i-1]}
      \pyplot@internal@add@ticks{internal axis=\@pyplot@which@ax, #1}
    }
  }{}
}
% Internal command that gets called to draw the ticks
\newcommand{\pyplot@internal@add@ticks}[1]{
  % ..:: Select axis ::..
  \setkeys{@pyplot@keys}{internal axis=,#1}
  \ifthenelse{\equal{\@pyplot@tick@internal@axis}{}}{
    \PackageError{pyplot}{Unspecified axis}{Set axis=x or axis=y}
  }{}

  % ..:: Min and max values that ticks span ::..
  \ifthenelse{\equal{\@pyplot@tick@internal@axis}{x}}{
    \setkeys{@pyplot@keys}{min=\@pyplot@plotarea@xmin,#1}
    \setkeys{@pyplot@keys}{max=\@pyplot@plotarea@xmax,#1}
  }{
    \setkeys{@pyplot@keys}{min=\@pyplot@plotarea@ymin,#1}
    \setkeys{@pyplot@keys}{max=\@pyplot@plotarea@ymax,#1}
  }

  % ..:: Other parameters ::..
  \setkeys{@pyplot@keys}{count=5,#1}
  \setkeys{@pyplot@keys}{tick size=0.06\@pyplot@unit,#1}
  \setkeys{@pyplot@keys}{label size=0.13,#1}
  \pyplot@internal@compute@scaling{\@pyplot@tick@label@size}
  \setkeys{@pyplot@keys}{label offset=0.1,#1}
  \setkeys{@pyplot@keys}{format=.2f,#1}

  % ..:: Set the tick label formatter ::..
  \pykzexec{pyplot_label_fmt = lambda x: '{0:\@pyplot@label@format}'.format(x)}
  \xdef\@pyplot@format@function{pyplot_label_fmt}

  \pykzexec{pyplot_tick_y = lambda x: 0.0}
  \foreach \@i in {1,...,\@pyplot@ticks@count} {

    % ..:: Draw the "tick" ::..
    \pykzmathinline[@pyplot@tick@value]{(\@i-1)/(\@pyplot@ticks@count-1)*(
      \@pyplot@ticks@max-\@pyplot@ticks@min)+\@pyplot@ticks@min}
    \ifthenelse{\equal{\@pyplot@tick@internal@axis}{x}}{
      \tikzmath{\@pyplot@tick@pos=\pyplot@internal@map@x{\@pyplot@tick@value};}
      \coordinate (tmp) at (\@pyplot@tick@pos, 0.0);
      \coordinate (pyplot tick pos) at (tmp|-pyplot origin);
      \draw[pyplot major tick] (pyplot tick pos) -- ++(0,-\@pyplot@ticks@size)
      node[pos=1](tick label coord){};
    }{
      \tikzmath{\@pyplot@tick@pos=\pyplot@internal@map@y{\@pyplot@tick@value};}
      \coordinate (tmp) at (0.0, \@pyplot@tick@pos);
      \coordinate (pyplot tick pos) at (tmp-|\@pyplot@y@axis@draw@ref@point);
      \draw[pyplot major tick] (pyplot tick pos) -- ++(-\@pyplot@ticks@size,0)
      node[pos=1](tick label coord){};
    }

    % ..:: Get tick label text (the number) ::..
    \pykzmathinline[@pyplot@tick@label]{
      \@pyplot@format@function(\@pyplot@tick@value)}
    \pykzmathinline[@pyplot@tick@label@abs]{
      \@pyplot@format@function(abs(\@pyplot@tick@label))}

    % ..:: Absolute-value tick label just for centering ::..
    \ifthenelse{\equal{\@pyplot@tick@internal@axis}{x}}{
      \def\@pyplot@internal@tick@label@anchor{north}
      \def\@pyplot@internal@tick@label@shift{
        0,-\@pyplot@tick@label@offset\@pyplot@unit}
    }{
      \def\@pyplot@internal@tick@label@anchor{east}
      \def\@pyplot@internal@tick@label@shift{
        -\@pyplot@tick@label@offset\@pyplot@unit,0}
    }
    \node[pyplot major tick label,
    scale=\@pyplot@label@scale,
    anchor=\@pyplot@internal@tick@label@anchor,
    shift={(\@pyplot@internal@tick@label@shift)},
    opacity=0] (pyplot tick label) at (tick label coord)
    {\phantom{$\@pyplot@tick@label@abs$}};

    \begin{pgfonlayer}{pyplot fg}
      % ..:: Background of the tick label ::..
      \fill[pyplot tick background]
      (pyplot tick label.south west) rectangle (pyplot tick label.north east);

      % ..:: The actual tick label ::..
      \node[pyplot major tick label,
      scale=\@pyplot@label@scale,
      anchor=north east] at (pyplot tick label.north east)
      {$\@pyplot@tick@label$};
    \end{pgfonlayer}

  }
}

% Create axis label
\define@key{@pyplot@keys}{label}{\def\@pyplot@axis@label@text{#1}}
\define@key{@pyplot@keys}{anchor}{\def\@pyplot@axis@label@anchor{#1}}
\define@key{@pyplot@keys}{xshift}{\def\@pyplot@axis@label@xshift{#1}}
\define@key{@pyplot@keys}{yshift}{\def\@pyplot@axis@label@yshift{#1}}
\define@key{@pyplot@keys}{rotate}{\def\@pyplot@axis@label@rotate{#1}}
\newcommand{\pyplot@axis@label}[1]{
  \setkeys{@pyplot@keys}{axis=,#1}
  \ifthenelse{\equal{\@pyplot@tick@axis}{}}{
    \PackageError{pyplot}{Unspecified axis}{Set axis=x or axis=y}
  }{}

  \setkeys{@pyplot@keys}{label=,#1}

  \ifthenelse{\equal{\@pyplot@tick@axis}{x}}{
    \setkeys{@pyplot@keys}{anchor=south east,#1}
    \setkeys{@pyplot@keys}{xshift=0,#1}
    \setkeys{@pyplot@keys}{yshift=0.15,#1}
  }{
    \setkeys{@pyplot@keys}{anchor=north east,#1}
    \setkeys{@pyplot@keys}{xshift=-0.15,#1}
    \setkeys{@pyplot@keys}{yshift=0,#1}
  }

  \setkeys{@pyplot@keys}{rotate=0,#1}
  \setkeys{@pyplot@keys}{label size=0.18,#1}
  \pyplot@internal@compute@scaling{\@pyplot@tick@label@size}

  % Draw the label
  \def\@pyplot@tmp@tip{ tip}
  \node[pyplot axis label,
  anchor=\@pyplot@axis@label@anchor,
  scale=\@pyplot@label@scale,
  xshift=\@pyplot@axis@label@xshift\@pyplot@unit,
  yshift=\@pyplot@axis@label@yshift\@pyplot@unit]
  at (pyplot \@pyplot@tick@axis\@pyplot@tmp@tip)
  {\@pyplot@axis@label@text};
}

% Map x value from math to TikZ coordinates
\newcommand{\pyplot@internal@map@x}[1]{
  ((#1-\@pyplot@plotarea@xmin)/(\@pyplot@plotarea@xmax-\@pyplot@plotarea@xmin)*
  (\@pyplot@plotarea@tikz@xmax-\@pyplot@plotarea@tikz@xmin)+
  \@pyplot@plotarea@tikz@xmin)
}

% Map y value from math to TikZ coordinates
\newcommand{\pyplot@internal@map@y}[1]{
  ((#1-\@pyplot@plotarea@ymin)/(\@pyplot@plotarea@ymax-\@pyplot@plotarea@ymin)*
  (\@pyplot@plotarea@tikz@ymax-\@pyplot@plotarea@tikz@ymin)+
  \@pyplot@plotarea@tikz@ymin)
}

% Generate a linear range of N points from min to max
\newcommand{\pyplot@linrange}[4]{
  \def\@pyplot@linrange@name{#1}
  \def\@pyplot@linrange@min{#2}
  \def\@pyplot@linrange@max{#3}
  \def\@pyplot@linrange@N{#4}
  \pykzexec{\@pyplot@linrange@name=np.linspace(\@pyplot@linrange@min,
    \@pyplot@linrange@max,\@pyplot@linrange@N)}
}

% Map a function over an x-range to generate (x,y) data.
\newcommand{\pyplot@map}[3][]{
  \def\@pyplot@xy@data@name{#1}
  \def\@pyplot@x@data{#2}
  \def\@pyplot@function{#3}

  % Determine max stride
  \pykzmathinline[@pyplot@x@data@size]{len(\@pyplot@x@data)}
  \pykzmathinline[@pyplot@unit@strlen]{len('\@pyplot@unit')}

  \tikzmath{
    \@pyplot@map@stride = int(floor(\@pyplot@python@maxchar/(
    2*(9+\@pyplot@map@precision+\@pyplot@unit@strlen)+4)));
    \@pyplot@stride@steps = int(\@pyplot@x@data@size/\@pyplot@map@stride)+1;
  }

  \def\@pyplot@exp{e}
  \pykzexec{pyplot_inbounds = lambda x, y:
    (not \@pyplot@clip) or
    (x>=\@pyplot@plotarea@xmin and
    x<=\@pyplot@plotarea@xmax and
    y>=\@pyplot@plotarea@ymin and
    y<=\@pyplot@plotarea@ymax)}

  \def\@pyplot@xy@data{}
  \foreach \@i in {1,...,\@pyplot@stride@steps} {
    \pykzmathinline[pyplot@xy@batch]{
      ' '.join('({0:.\@pyplot@map@precision\@pyplot@exp}*1\@pyplot@unit,%
      {1:.\@pyplot@map@precision\@pyplot@exp}*1\@pyplot@unit)'.format(
      \pyplot@internal@map@x{_x},
      \pyplot@internal@map@y{\@pyplot@function(_x)}
      ) if pyplot_inbounds(_x, \@pyplot@function(_x)) else ''
      for _x in
      \@pyplot@x@data[(\@i-1)*\@pyplot@map@stride:min(\@i*\@pyplot@map@stride,
      \@pyplot@x@data@size)])}
    \xdef\@pyplot@xy@data{\@pyplot@xy@data \pyplot@xy@batch}
  }

  \ifthenelse{\equal{\@pyplot@xy@data@name}{}}{}{
    \expandafter\xdef\csname \@pyplot@xy@data@name\endcsname{\@pyplot@xy@data}}%
}

% Create a callable colormap object cmap(x)
\define@key{@pyplot@keys}{map}{\def\@pyplot@colormap@style{#1}}
\define@key{@pyplot@keys}{name}{\def\@pyplot@colormap@function@name{#1}}
\newcommand{\pyplot@colormap}[3][]{
  \setkeys{@pyplot@keys}{map=inferno,#1}
  \setkeys{@pyplot@keys}{name=@cmap,#1}
  \def\@pyplot@colormap@min{#2}
  \def\@pyplot@colormap@max{#3}
  \pykzexec{import matplotlib as mpl}
  \pykzexec{from matplotlib import pyplot as plt}
  \pykzexec{cmap_base = plt.get_cmap('\@pyplot@colormap@style')}
  \pykzexec{cmap_nrm = mpl.colors.Normalize(vmin=\@pyplot@colormap@min,
    vmax=\@pyplot@colormap@max)}
  \pykzexec{cmap_rgba = mpl.cm.ScalarMappable(norm=cmap_nrm, cmap=cmap_base)}
  \pykzexec{cmap = lambda x : ('{0}'.format(mpl.colors.to_hex(
    cmap_rgba.to_rgba(x))))[1:]}
  % Make a function which you can call to generate a new color
  \expandafter\newcommand\csname\@pyplot@colormap@function@name\endcsname[2][]{
    \pykzmathinline[@pyplot@colormap@clr]{cmap(##2)}
    \definecolor{##1}{HTML}{\@pyplot@colormap@clr}
  }
}

% Create a clipped plot area within which elemnts are not draw if they fall
% outside the plot area
\define@key{@pyplot@keys}{pad}{\def\@pyplot@clip@area@pad{#1}}
\define@key{@pyplot@keys}{leftpad}{\def\@pyplot@clip@area@pad@left{#1}}
\newenvironment{pyplot@plot@area}[1][]{%
  \setkeys{@pyplot@keys}{pad=0,#1}
  \setkeys{@pyplot@keys}{leftpad=-1,#1}
  \begin{scope}%
    % General clipping area (apply global padding)
    \coordinate (clip area bottom left) at
    ($(pyplot plotarea bottom left)+(-\@pyplot@clip@area@pad,
    -\@pyplot@clip@area@pad)$);
    \coordinate (clip area top right) at
    ($(pyplot plotarea top right)+(\@pyplot@clip@area@pad,
    \@pyplot@clip@area@pad)$);

    % Apply per-edge padding
    \pgfmathparse{\@pyplot@clip@area@pad@left<0 ? 1:0}
    \ifthenelse{\pgfmathresult>0}{}{
      \coordinate (clip area bottom left) at
      ($(clip area bottom left)+(\@pyplot@clip@area@pad-
      \@pyplot@clip@area@pad@left,0)$);
    }

    % Make clipping area
    \clip (clip area bottom left) rectangle (clip area top right);
  }{%
  \end{scope}%
}

%%%%%%%%%%%%%%%%%%%%%%%%%%%%%%%%%%%%%%%
% Private functions
%%%%%%%%%%%%%%%%%%%%%%%%%%%%%%%%%%%%%%%

% TODO

\makeatother

%%% Local Variables:
%%% mode: latex
%%% TeX-master: t
%%% End:

}

\NewEnviron{@pie@shell}{%
  \immediate\write18{\BODY}%
}

\newenvironment{pykzpicture}[1][]{%
  % ---------------------------------------------------------
  %
  % A wrapper of the tikzpicture environment, which does all the PykZ
  % initialization in the background
  %
  % Parameters
  % ----------
  % #1 : same options as you would pass tikzpicture
  % #2 : relative location of PykZ directory
  %
  % ---------------------------------------------------------
  %
  % -------------------------------------
  % Check if 3D required
  % -------------------------------------
  \edef\@threed{3d}%
  \pykzmathinline[@use@threed]{'\@threed' in str('#1')}%
  \pykzmathinline[@tikz@opts]{str('#1').replace('\@threed,','').
    replace(',\@threed','').replace('\@threed','')}%
  % -------------------------------------
  % Start the tikzpicture
  % -------------------------------------
  \tikzset{%
    apply style/.code={%
      \tikzset{#1}%
    }%
  }%
  \begin{tikzpicture}[
    % Rounded line ends
    line cap=round,
    % Bevelled (non-sharp) polygon corners
    line join=bevel,
    % User commands
    apply style/.expand once=\@tikz@opts]
    %
    % -------------------------------------
    % Common styles
    % -------------------------------------
    \tikzset
    {
      % No default separation
      every node/.append style=
      {
        inner sep=0,
        outer sep=0
      },
      % Default arrow
      arrow/.append style=
      {
        -latex
      },
      % Text labels
      label/.append style=
      {
        outer sep=1
      },
      % Bezier curve controls style
      bezier/.style={
        postaction={
          decoration={
            show path construction,
            curveto code={
              \draw [blue]
              (\tikzinputsegmentfirst) -- (\tikzinputsegmentsupporta)
              (\tikzinputsegmentlast) -- (\tikzinputsegmentsupportb);
              \fill [red, opacity=0.5]
              (\tikzinputsegmentsupporta) circle [radius=.5ex]
              (\tikzinputsegmentsupportb) circle [radius=.5ex];
            }
          },
          decorate
        }}
    }%
    %
    % -------------------------------------
    % Activate 3D plot
    % -------------------------------------
    \ifthenelse{\equal{\@use@threed}{True}}{
      \xdef\@piv@td@main@alt{60}%
      \xdef\@piv@td@main@azi{125}%
      \tdplotsetmaincoords{\@piv@td@main@alt}{\@piv@td@main@azi}%
      \pykzmathinline[@piv@td@R]{(np.eye(3)).tolist()}%
    }{}
    \pykzmathinline[@threed@cmd]{'tdplot_main_coords' if (\@use@threed) else ''}%
    \begin{scope}[\@threed@cmd]%
      %
      % -------------------------------------
      % User code
      % -------------------------------------
    }
    % BODY
    {
    \end{scope}%
  \end{tikzpicture}%
}

%%%%%%%%%%%%%%%%%%%%%%%%%%%%%%%
% Math library
%%%%%%%%%%%%%%%%%%%%%%%%%%%%%%%

\newcommand{\pykzexec}[1]{%
  % ---------------------------------------------------------
  %
  % Call a function in pykz_custom.py, and store result in
  % local variables of Python session.
  %
  % ---------------------------------------------------------
\begin{@pie@shell}
cd \@piv@relpath && ./client "<exec> #1"
\end{@pie@shell}%
}

\newcommand{\pykzmathinline}[2][]{%
  % ---------------------------------------------------------
  %
  % Send one-line math command to Python, and store result in \out.
  %
  % ---------------------------------------------------------
\begin{@pie@shell}%
curdir=$(pwd) $() &&
cd \@piv@relpath &&
./client "#2" &&
mv result.tex $curdir $()
\end{@pie@shell}%
  \input{result.tex}%
  \ifthenelse{\equal{\out}{PYKZ_FAIL}}{%
    \PackageError{PykZ}{Python failed to process your command}{%
      Is your command proper Python syntax?}%
  }{}%
  \ifthenelse{\equal{#1}{}}{}{\expandafter\xdef\csname #1\endcsname{\out}}%
}

\newenvironment{pykzmathblock}
% ---------------------------------------------------------
%
% Save a block of Python code in a file accessible to PykZ
% for later execution.
%
% ---------------------------------------------------------
{\VerbatimOut{pykz_custom.py}}
{\endVerbatimOut
\begin{@pie@shell}
mv pykz_custom.py \@piv@pypath
\end{@pie@shell}
}

\newcommand{\pykzmathfunction}[1]{
  % ---------------------------------------------------------
  %
  % Call a function in pykz_custom.py, and store result in
  % local variables of Python session.
  %
  % ---------------------------------------------------------
\begin{@pie@shell}
cd \@piv@relpath && ./client "<function> #1"
\end{@pie@shell}
}

%%%%%%%%%%%%%%%%%%%%%%%%%%%%%%%
% Drawing
%%%%%%%%%%%%%%%%%%%%%%%%%%%%%%%

\newcommand{\shift}[2]{%
  %
  % Shift a point #1 by amount #2.
  %
  ($(#1)+(#2)$)%
}

\renewcommand{\bezier}[2]{%
  % ---------------------------------------------------------
  %
  % Bezier curve controls.
  %
  % Parameters
  % ----------
  % #1 : angle:radius for start point control
  % #2 : angle:radius for end point control
  %
  % ---------------------------------------------------------
  .. controls ++(#1) and ++(#2) ..}

%%%%%%%%%%%%%%%%%%%%%%%%%%%%%%%
% Beamer tools
%%%%%%%%%%%%%%%%%%%%%%%%%%%%%%%

\newenvironment{pageplace}[1]
{
  \xdef\@pos@args{#1}
  \pykzpicture[remember picture,
  overlay,
  shift={(current page.center)}]
  \pykzmathinline[@arg@length]{len([\@pos@args])}
  \ifthenelse{\equal{\@arg@length}{2}}{
    % Direct position
    \pykzmathinline[@pos@x]{str([\@pos@args][0]).replace(chr(39),'')}
    \pykzmathinline[@pos@y]{str([\@pos@args][1]).replace(chr(39),'')}
    \dimensionalize{\@pos@x,\@pos@y}
  }{
    % Position using grid
    \tikzmath{
      \@page@width = \paperwidth;
      \@page@height = \paperheight;
    }
    \pykzmathinline[@pos@x]{(([\@pos@args][3]-0.5)/[\@pos@args][1]-0.5)*\@page@width}
    \pykzmathinline[@pos@y]{(-([\@pos@args][2]-0.5)/[\@pos@args][0]+0.5)*\@page@height}
    \dimensionalize{\@pos@x,\@pos@y}
  }
  \coordinate (loc) at (\@pos@x,\@pos@y);
}{
  \endpykzpicture
}

%%%%%%%%%%%%%%%%%%%%%%%%%%%%%%%
% Utilities
%%%%%%%%%%%%%%%%%%%%%%%%%%%%%%%

\def\convertto#1#2{\strip@pt\dimexpr #2*65536/\number\dimexpr 1#1}
\define@key{@pik@keys}{unit}{\def\@@@unit{#1}}
\newcommand{\dimensionalize}[2][]{%
  %
  % Append to #1 the unit \@@@unit.
  % Common use case is to append "pt" to a tikzmath output.
  %
  % -- Parameters
  \setkeys{@pik@keys}{unit=pt,#1}%
  % -- Function body
  \foreach \@@var in {#2} {%
    \ifthenelse{\equal{\@@@unit}{}}{%
      \pykzmathinline{'\@@var'[:-2]}%
      \expandafter\xdef\@@var{\out}%
    }{%
      \tikzmath{\@@@@value=\@@var;}% % to pt
      \def\@pt@unit{pt}%
      \xdef\@@@@tmp{\@@@@value\@pt@unit}% % append "pt" unit
      \expandafter\xdef\@@var{\convertto{\@@@unit}{\@@@@tmp}\@@@unit}% % convert to target unit
    }%
  }%
}

\define@key{@pik@keys}{amount}{\def\@@amount{#1}}
\newcommand{\counter@increment}[2][]{%
  %
  % Increments counter #2 by increment amount #1 (default: 1).
  %
  % -- Parameters
  \setkeys{@pik@keys}{amount=1,#1}%
  % -- Function body
  \tikzmath{\@@counter@next=int(#2+\@@amount);}%
  \xdef#2{\@@counter@next}%
}

\makeatother

%%% Local Variables:
%%% mode: latex
%%% TeX-master: t
%%% End:
