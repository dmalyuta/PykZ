%%%%%%%%%%%%%%%%%%%%%%%%%%%%%%%%%%%%%%%%%%%%%%%%%%%%%%%%%%%%%%%%%%%%%%%%
%
% PykZ
%
% A package that combines the speed of Python for numerical calculation and the
% drawing capabilities of TikZ, to make publication-quality drawing faster.
%
% Some conventions:
%
%   \@pik@<name> : PykZ internal key <name>
%   \@piv@<name> : PykZ internal variable <name>
%   \@pic@<name> : PykZ internal command <name>
%   @pie@<name> : PykZ internal environment <name>
%
% Danylo Malyuta, 2020
% 
%%%%%%%%%%%%%%%%%%%%%%%%%%%%%%%%%%%%%%%%%%%%%%%%%%%%%%%%%%%%%%%%%%%%%%%%

\makeatletter

%%%%%%%%%%%%%%%%%%%%%%%%%%%%%%%
% Load packages
%%%%%%%%%%%%%%%%%%%%%%%%%%%%%%%

\usepackage{tikz}
\usepackage{tikz-3dplot}
\usepackage{pythontex}

%%%%%%%%%%%%%%%%%%%%%%%%%%%%%%%
% Load TikZ libraries
%%%%%%%%%%%%%%%%%%%%%%%%%%%%%%%

\usetikzlibrary{arrows}
\usetikzlibrary{intersections}
\usetikzlibrary{decorations.markings}
\usetikzlibrary{math}
\usetikzlibrary{shapes}
% \usetikzlibrary{calc}
% \usetikzlibrary{tikzmark}
% \usetikzlibrary{positioning}
% \usetikzlibrary{scopes}
% \usetikzlibrary{shapes.multipart}
% \usetikzlibrary{shapes.geometric}
% \usetikzlibrary{bending}
% \usetikzlibrary{decorations.text}

%%%%%%%%%%%%%%%%%%%%%%%%%%%%%%%
% General functions
%%%%%%%%%%%%%%%%%%%%%%%%%%%%%%%

\gdef\@piv@python@delay{0.5} % Startup wait for Python server

\define@key{@pik@keys}{relpath}{\def\@piv@relpath{#1}}
\newcommand{\pykzinit}[1][]{
  % ---------------------------------------------------------
  %
  % Initialization of PykZ.
  % Starts the Python background server process.
  %
  % ---------------------------------------------------------
  \setkeys{@pik@keys}{relpath=.,#1}
  \xdef\@piv@relpath{\@piv@relpath}
  % Start a background Python server
\begin{@pie@shell}
curdir=$(pwd) &&
cd \@piv@relpath &&
python server.py &
\end{@pie@shell}
  % Allow some start-up time for Python
\begin{@pie@shell}
sleep \@piv@python@delay
\end{@pie@shell}
}

\NewEnviron{@pie@shell}{
  \immediate\write18{\BODY}
}

\NewEnviron{pykzpicture}[2][]{
  % ---------------------------------------------------------
  %
  % A wrapper of the tikzpicture environment, which does all the PykZ
  % initialization in the background
  %
  % Parameters
  % ----------
  % #1 : same options as you would pass tikzpicture
  % #2 : relative location of PykZ directory
  %
  % ---------------------------------------------------------

  \pykzinit[relpath=#2] % Initialize PykZ

  % -------------------------------------
  % 3D coordinate initial angles
  % -------------------------------------
  \xdef\@piv@td@main@alt{0}
  \xdef\@piv@td@main@azi{0}
  \tdplotsetmaincoords{\@piv@td@main@alt}{\@piv@td@main@azi}
  \pykzmath{(np.eye(3)).tolist()}
  \xdef\@piv@td@R{\out}
  
  \begin{tikzpicture}[
    line cap=round,
    line join=bevel,
    #1]
    
    % -------------------------------------
    % Common styles
    %-------------------------------------
    \tikzset
    {
      every node/.append style=
      {
        inner sep=0,
        outer sep=0
      },
      arrow/.append style=
      {
        -latex
      },
      label/.append style=
      {
        outer sep=1
      }
    }

    % -------------------------------------
    % User code
    %-------------------------------------
    \BODY
  \end{tikzpicture}
}

%%%%%%%%%%%%%%%%%%%%%%%%%%%%%%%
% Math library
%%%%%%%%%%%%%%%%%%%%%%%%%%%%%%%

\newcommand{\pykzmath}[1]{
  % ---------------------------------------------------------
  %
  % Send math command to Python, and store result in \out.
  %
  % ---------------------------------------------------------  
\begin{@pie@shell}
curdir=$(pwd) &&
cd \@piv@relpath &&
./client "#1" &&
mv result.tex $curdir
\end{@pie@shell}
  \input{result.tex}
}

%%%%%%%%%%%%%%%%%%%%%%%%%%%%%%%
% Utilities
%%%%%%%%%%%%%%%%%%%%%%%%%%%%%%%


\def\convertto#1#2{\strip@pt\dimexpr #2*65536/\number\dimexpr 1#1}
\define@key{@dimensionalize@keys}{unit}{\def\@@@unit{#1}}
\newcommand{\dimensionalize}[2][]{%
  %
  % Append to #1 the unit \@@@unit.
  % Common use case is to append "pt" to a tikzmath output.
  %
  % -- Parameters
  \setkeys{@dimensionalize@keys}{unit=pt,#1}%
  % -- Function body
  \foreach \@@var in {#2} {
    \tikzmath{\@@@@value=\@@var;} % to pt
    \def\@pt@unit{pt}%
    \xdef\@@@@tmp{\@@@@value\@pt@unit} % append "pt" unit
    \expandafter\xdef\@@var{\convertto{\@@@unit}{\@@@@tmp}\@@@unit} % convert to target unit
  }
}

\makeatother

%%% Local Variables:
%%% mode: latex
%%% TeX-master: t
%%% End:
