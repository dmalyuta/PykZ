%%%%%%%%%%%%%%%%%%%%%%%%%%%%%%%%%%%%%%%%%%%%%%%%%%%%%%%%%%%%%%%%%%%%%%%%
%
% PykZ
%
% A package that combines the speed of Python for numerical calculation and the
% drawing capabilities of TikZ, to make publication-quality drawing faster.
%
% Some conventions:
%
%   \@pik@<name> : PykZ internal key <name>
%   \@piv@<name> : PykZ internal variable <name>
%   \@pic@<name> : PykZ internal command <name>
%   @pie@<name> : PykZ internal environment <name>
%
% Danylo Malyuta, 2020
% 
%%%%%%%%%%%%%%%%%%%%%%%%%%%%%%%%%%%%%%%%%%%%%%%%%%%%%%%%%%%%%%%%%%%%%%%%

\makeatletter

%%%%%%%%%%%%%%%%%%%%%%%%%%%%%%%
% Load packages
%%%%%%%%%%%%%%%%%%%%%%%%%%%%%%%

\usepackage{tikz}
\usepackage{tikz-3dplot}
\usepackage{pythontex}

%%%%%%%%%%%%%%%%%%%%%%%%%%%%%%%
% Load TikZ libraries
%%%%%%%%%%%%%%%%%%%%%%%%%%%%%%%

\usetikzlibrary{arrows}
\usetikzlibrary{intersections}
\usetikzlibrary{decorations.markings}
% \usetikzlibrary{math}
% \usetikzlibrary{calc}
% \usetikzlibrary{shapes}
% \usetikzlibrary{tikzmark}
% \usetikzlibrary{positioning}
% \usetikzlibrary{scopes}
% \usetikzlibrary{shapes.multipart}
% \usetikzlibrary{shapes.geometric}
% \usetikzlibrary{bending}
% \usetikzlibrary{decorations.text}

%%%%%%%%%%%%%%%%%%%%%%%%%%%%%%%
% General functions
%%%%%%%%%%%%%%%%%%%%%%%%%%%%%%%

\define@key{@pik@keys}{relpath}{\def\@piv@relpath{#1}}
\newcommand{\pykz@init}[1][]{
  \setkeys{@pik@keys}{relpath=.,#1}
  \xdef\@piv@relpath{\@piv@relpath}
  % TODO start a Python server running in background
}

%%%%%%%%%%%%%%%%%%%%%%%%%%%%%%%
% Math library
%%%%%%%%%%%%%%%%%%%%%%%%%%%%%%%

\NewEnviron{@pie@shell}{
  \immediate\write18{\BODY}
}

\define@key{@pik@keys}{out}{\def\@piv@out{#1}}
\newcommand{\pykz@matrix@multiply}[3][]{
  % -----------------------------------------------
  % Matrix multiplication #2*#3.
  %
  % Parameters
  % ----------
  % #1 : parameters (out: what to name the result)
  % #2 : left matrix
  % #3 : right matrix
  % -----------------------------------------------
  \setkeys{@pik@keys}{out=out,#1}
  \xdef\@piv@out{"\@piv@out"}
\begin{@pie@shell}
curdir=$(pwd) &&
cd \@piv@relpath &&
python -c 'import functions as f;
f.matrix_multiply(#2,#3,out=\@piv@out)' > $curdir/tmp.tex
\end{@pie@shell}
% \begin{@pie@shell}
% curdir=$(pwd) &&
% cd \@piv@relpath &&
% echo Hello &&
% python -c 'import functions as f; print("hello")' > $curdir/tmp.tex
% \end{@pie@shell}
  \input{tmp.tex}
}

\newcommand{\pykz@matrix@inverse}[2][]{
  % -----------------------------------------------
  % Invert matrix #2.
  %
  % Parameters
  % ----------
  % #1 : parameters (out: what to name the result)
  % #2 : matrix to be inverted
  % -----------------------------------------------
  \setkeys{@pik@keys}{out=out,#1}
  \xdef\@piv@out{"\@piv@out"}
\begin{@pie@shell}
curdir=$(pwd) &&
cd \@piv@relpath &&
python -c 'import functions as f;
f.matrix_inverse(#2,out=\@piv@out)' > $curdir/tmp.tex
\end{@pie@shell}
  \input{tmp.tex}
}

\newcommand{\pykz@matrix@add}[3][]{
  % -----------------------------------------------
  % Add matrices #2+#3.
  %
  % Parameters
  % ----------
  % #1 : parameters (out: what to name the result)
  % #2 : left matrix
  % #3 : right matrix
  % -----------------------------------------------
  \setkeys{@pik@keys}{out=out,#1}
  \xdef\@piv@out{"\@piv@out"}
\begin{@pie@shell}
curdir=$(pwd) &&
cd \@piv@relpath &&
python -c 'import functions as f;
f.matrix_add(#2,#3,out=\@piv@out)' > $curdir/tmp.tex
\end{@pie@shell}
  \input{tmp.tex}
}

\makeatother

%%% Local Variables:
%%% mode: latex
%%% TeX-master: t
%%% End:
