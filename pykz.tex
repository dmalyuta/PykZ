%%%%%%%%%%%%%%%%%%%%%%%%%%%%%%%%%%%%%%%%%%%%%%%%%%%%%%%%%%%%%%%%%%%%%%%%
%
% PykZ
%
% A package that combines the speed of Python for numerical calculation and the
% drawing capabilities of TikZ, to make publication-quality drawing faster.
%
% Some conventions:
%
%   \@pik@<name> : PykZ internal key <name>
%   \@piv@<name> : PykZ internal variable <name>
%   \@pic@<name> : PykZ internal command <name>
%   @pie@<name> : PykZ internal environment <name>
%
% Danylo Malyuta, 2020
% 
%%%%%%%%%%%%%%%%%%%%%%%%%%%%%%%%%%%%%%%%%%%%%%%%%%%%%%%%%%%%%%%%%%%%%%%%

\makeatletter

%%%%%%%%%%%%%%%%%%%%%%%%%%%%%%%
% Load packages
%%%%%%%%%%%%%%%%%%%%%%%%%%%%%%%

\usepackage{tikz}
\usepackage{tikz-3dplot}
\usepackage{pythontex}
\usepackage{filecontents}

%%%%%%%%%%%%%%%%%%%%%%%%%%%%%%%
% Load TikZ libraries
%%%%%%%%%%%%%%%%%%%%%%%%%%%%%%%

\usetikzlibrary{backgrounds}
\usetikzlibrary{arrows}
\usetikzlibrary{arrows.meta}
\usetikzlibrary{intersections}
\usetikzlibrary{math}
\usetikzlibrary{shapes}
\usetikzlibrary{positioning}
\usetikzlibrary{decorations.markings}
\usetikzlibrary{decorations.pathreplacing}
\usetikzlibrary{tikzmark}
\usetikzlibrary{fadings}
% \usetikzlibrary{calc}
% \usetikzlibrary{tikzmark}
% \usetikzlibrary{scopes}
% \usetikzlibrary{shapes.multipart}
% \usetikzlibrary{shapes.geometric}
% \usetikzlibrary{bending}
% \usetikzlibrary{decorations.text}

%%%%%%%%%%%%%%%%%%%%%%%%%%%%%%%
% General functions
%%%%%%%%%%%%%%%%%%%%%%%%%%%%%%%

\gdef\@piv@python@delay{0.5} % Startup wait for Python server

\newcommand{\pykzinit}[1]{
  % ---------------------------------------------------------
  %
  % Initialization of PykZ.
  % Starts the Python background server process.
  % Must be called before doing anything that uses PykZ.
  %
  % ---------------------------------------------------------
  \xdef\@piv@relpath{#1}
  % Start a background Python server
\begin{@pie@shell}
curdir=$(pwd) &&
cd \@piv@relpath &&
python server.py &
\end{@pie@shell}
  % Allow some start-up time for Python
\begin{@pie@shell}
sleep \@piv@python@delay
\end{@pie@shell}
  % Determine path to directory where latex_common resides
  \pykzmathinline[@piv@pypath]{'\@piv@relpath'[:'\@piv@relpath'.find('latex_common')]}
}

\NewEnviron{@pie@shell}{%
  \immediate\write18{\BODY}%
}

\newenvironment{pykzpicture}[1][]{%
  % ---------------------------------------------------------
  % 
  % A wrapper of the tikzpicture environment, which does all the PykZ
  % initialization in the background
  % 
  % Parameters
  % ----------
  % #1 : same options as you would pass tikzpicture
  % #2 : relative location of PykZ directory
  % 
  % ---------------------------------------------------------
  % 
  % -------------------------------------
  % Check if 3D required
  % -------------------------------------
  \edef\@threed{3d}%
  \pykzmathinline[@use@threed]{'\@threed' in str('#1')}%
  \pykzmathinline[@tikz@opts]{str('#1').replace('\@threed,','').
    replace(',\@threed','').replace('\@threed','')}%
  % -------------------------------------
  % Start the tikzpicture
  % -------------------------------------
  \tikzset{%
    apply style/.code={%
      \tikzset{#1}%
    }%
  }%
  \begin{tikzpicture}[
    % Rounded line ends
    line cap=round,
    % Bevelled (non-sharp) polygon corners
    line join=bevel,
    % User commands
    apply style/.expand once=\@tikz@opts]
    % 
    % -------------------------------------
    % Common styles
    % -------------------------------------
    \tikzset
    {
      % No default separation
      every node/.append style=
      {
        inner sep=0,
        outer sep=0
      },
      % Default arrow
      arrow/.append style=
      {
        -latex
      },
      % Text labels
      label/.append style=
      {
        outer sep=1
      },
      % Bezier curve controls style
      bezier/.style={
        postaction={
          decoration={
            show path construction,
            curveto code={
              \draw [blue] 
              (\tikzinputsegmentfirst) -- (\tikzinputsegmentsupporta)
              (\tikzinputsegmentlast) -- (\tikzinputsegmentsupportb);
              \fill [red, opacity=0.5] 
              (\tikzinputsegmentsupporta) circle [radius=.5ex]
              (\tikzinputsegmentsupportb) circle [radius=.5ex];
            }
          },
          decorate
        }}
    }%
    % 
    % -------------------------------------
    % Activate 3D plot
    % -------------------------------------
    \ifthenelse{\equal{\@use@threed}{True}}{
      \xdef\@piv@td@main@alt{60}%
      \xdef\@piv@td@main@azi{125}%
      \tdplotsetmaincoords{\@piv@td@main@alt}{\@piv@td@main@azi}%
      \pykzmathinline[@piv@td@R]{(np.eye(3)).tolist()}%
    }{}
    \pykzmathinline[@threed@cmd]{'tdplot_main_coords' if (\@use@threed) else ''}%
    \begin{scope}[\@threed@cmd]%
      % 
      % -------------------------------------
      % User code
      % -------------------------------------
    }
    % BODY
    {
    \end{scope}%
  \end{tikzpicture}%
}

%%%%%%%%%%%%%%%%%%%%%%%%%%%%%%%
% Math library
%%%%%%%%%%%%%%%%%%%%%%%%%%%%%%%

\newcommand{\pykzexec}[1]{%
  % ---------------------------------------------------------
  %
  % Call a function in pykz_custom.py, and store result in
  % local variables of Python session.
  %
  % ---------------------------------------------------------  
\begin{@pie@shell}
cd \@piv@relpath && ./client "<exec> #1"
\end{@pie@shell}%
}

\newcommand{\pykzmathinline}[2][]{%
  % ---------------------------------------------------------
  %
  % Send one-line math command to Python, and store result in \out.
  %
  % ---------------------------------------------------------  
\begin{@pie@shell}%
curdir=$(pwd) &&
cd \@piv@relpath &&
./client "#2" &&
mv result.tex $curdir
\end{@pie@shell}%
  \input{result.tex}%
  \ifthenelse{\equal{\out}{PYKZ_FAIL}}{%
    \PackageError{PykZ}{Python failed to process your command}{%
      Is your command proper Python syntax?}%
  }{}%
  \ifthenelse{\equal{#1}{}}{}{\expandafter\xdef\csname #1\endcsname{\out}}%
}

\newenvironment{pykzmathblock}
% ---------------------------------------------------------
% 
% Save a block of Python code in a file accessible to PykZ
% for later execution.
% 
% ---------------------------------------------------------  
{\VerbatimOut{pykz_custom.py}}
{\endVerbatimOut
\begin{@pie@shell}
mv pykz_custom.py \@piv@pypath
\end{@pie@shell}
}

\newcommand{\pykzmathfunction}[1]{
  % ---------------------------------------------------------
  %
  % Call a function in pykz_custom.py, and store result in
  % local variables of Python session.
  %
  % ---------------------------------------------------------  
\begin{@pie@shell}
cd \@piv@relpath && ./client "<function> #1"
\end{@pie@shell}
}

%%%%%%%%%%%%%%%%%%%%%%%%%%%%%%%
% Drawing
%%%%%%%%%%%%%%%%%%%%%%%%%%%%%%%

\renewcommand{\bezier}[2]{%
  % ---------------------------------------------------------
  %
  % Bezier curve controls.
  %
  % Parameters
  % ----------
  % #1 : angle:radius for start point control
  % #2 : angle:radius for end point control
  %
  % ---------------------------------------------------------  
  .. controls ++(#1) and ++(#2) ..}

%%%%%%%%%%%%%%%%%%%%%%%%%%%%%%%
% Beamer tools
%%%%%%%%%%%%%%%%%%%%%%%%%%%%%%%

\newenvironment{pageplace}[1]
{
  \xdef\@pos@args{#1}
  \pykzpicture[remember picture,
  overlay,
  shift={(current page.center)}]
  \pykzmathinline[@arg@length]{len([\@pos@args])}
  \ifthenelse{\equal{\@arg@length}{2}}{
    % Direct position
    \pykzmathinline[@pos@x]{str([\@pos@args][0]).replace(chr(39),'')}
    \pykzmathinline[@pos@y]{str([\@pos@args][1]).replace(chr(39),'')}
    \dimensionalize{\@pos@x,\@pos@y}
  }{
    % Position using grid
    \tikzmath{
      \@page@width = \paperwidth;
      \@page@height = \paperheight;
    }
    \pykzmathinline[@pos@x]{(([\@pos@args][3]-0.5)/[\@pos@args][1]-0.5)*\@page@width}
    \pykzmathinline[@pos@y]{(-([\@pos@args][2]-0.5)/[\@pos@args][0]+0.5)*\@page@height}
    \dimensionalize{\@pos@x,\@pos@y}
  }
  \coordinate (loc) at (\@pos@x,\@pos@y);
}{
  \endpykzpicture
}

%%%%%%%%%%%%%%%%%%%%%%%%%%%%%%%
% Drawing helpers
%%%%%%%%%%%%%%%%%%%%%%%%%%%%%%%

\newcommand{\shift}[2]{%
  %
  % Shift a point #1 by amount #2.
  %
  ($(#1)+(#2)$)%
}

%%%%%%%%%%%%%%%%%%%%%%%%%%%%%%%
% Utilities
%%%%%%%%%%%%%%%%%%%%%%%%%%%%%%%

\def\convertto#1#2{\strip@pt\dimexpr #2*65536/\number\dimexpr 1#1}
\define@key{@pik@keys}{unit}{\def\@@@unit{#1}}
\newcommand{\dimensionalize}[2][]{%
  %
  % Append to #1 the unit \@@@unit.
  % Common use case is to append "pt" to a tikzmath output.
  %
  % -- Parameters
  \setkeys{@pik@keys}{unit=pt,#1}%
  % -- Function body
  \foreach \@@var in {#2} {%
    \ifthenelse{\equal{\@@@unit}{}}{%
      \pykzmathinline{'\@@var'[:-2]}%
      \expandafter\xdef\@@var{\out}%
    }{%
      \tikzmath{\@@@@value=\@@var;}% % to pt
      \def\@pt@unit{pt}%
      \xdef\@@@@tmp{\@@@@value\@pt@unit}% % append "pt" unit
      \expandafter\xdef\@@var{\convertto{\@@@unit}{\@@@@tmp}\@@@unit}% % convert to target unit
    }%
  }%
}

\define@key{@pik@keys}{amount}{\def\@@amount{#1}}
\newcommand{\counter@increment}[2][]{%
  %
  % Increments counter #2 by increment amount #1 (default: 1).
  %
  % -- Parameters
  \setkeys{@pik@keys}{amount=1,#1}%
  % -- Function body
  \tikzmath{\@@counter@next=int(#2+\@@amount);}%
  \xdef#2{\@@counter@next}%
}

\makeatother

%%% Local Variables:
%%% mode: latex
%%% TeX-master: t
%%% End:
